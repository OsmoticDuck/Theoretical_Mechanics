\documentclass[11pt,a4paper]{article}
\usepackage[utf8]{inputenc}
\usepackage[english]{babel}
\usepackage{fontspec}
\usepackage{xunicode}
\usepackage{xltxtra}
\usepackage{polyglossia}
\usepackage{amsmath}
\usepackage{amsfonts}
\usepackage{amssymb}
\usepackage{graphicx}
\usepackage[left=2cm,right=2cm,top=2cm,bottom=2cm]{geometry}
\usepackage{afterpage}
\usepackage{xcolor}
\usepackage{pagecolor}
\usepackage{mathtools}
%\usepackage{fancyhdr}
\usepackage{hyperref}
\usepackage{blindtext}
\usepackage{mahjong}
\usepackage{amsthm}
\usepackage{chngcntr}
\usepackage{tikz}
\usepackage{bbold}
\usepackage{pdfpages}
\usepackage{xfrac}

\usepackage{stmaryrd}
\let\stmaryrdLightning\lightning

\author{Zehao Gao}

\begin{document}

\section*{Homework 4 solutions}

\paragraph{Question 1}

\begin{enumerate}

\item[(a)]

Conservation of momentum in \textit{lab frame}
\begin{align}
p_1+p_2=p_1'+p_2'=P_l
\end{align}

Conservation of energy in \textit{lab frame}
\begin{align}
\frac{1}{2}p_1v_1+\frac{1}{2}p_2v_2=\frac{1}{2}p_1'v_1'+\frac{1}{2}p_2'v_2'+Q=E
\end{align}
\begin{align}
\frac{1}{2m_1}p_1^2+\frac{1}{2m_2}p_2^2=\frac{1}{2m_1}{p'}_1^2+\frac{1}{2m_2}{p'}_2^2+Q=E
\end{align}

Conservation of momentum in \textit{zero momentum frame}
\begin{align}
p_a+p_b&=p_a'+p_b'=0 \\
p_a&=-p_b \\
p_a'&=-p_b'
\end{align}

Conservation of energy in \textit{zero momentum frame}
\begin{align}
\frac{1}{2}p_av_a+\frac{1}{2}p_bv_b=\frac{1}{2}p_a'v_a'+\frac{1}{2}p_b'v_b'+Q=E
\end{align}

Position of two particles in \textit{lab frame} are
\begin{align}
r_1,r_2
\end{align}

Position of two particles in \textit{zero momentum frame} are
\begin{align}
r_a&=r_1-r_s \\
r_b&=r_2-r_s
\end{align}

where $r_s$ is the position of center of mass. It is given by
\begin{align}
r_s=\frac{m_1r_1+m_2r_2}{m_1+m_2}
\end{align}

therefore we get
\begin{align}
r_a&=r_1-\frac{m_1r_1+m_2r_2}{m_1+m_2}=\frac{(m_1+m_2)r_1-m_1r_1-m_2r_2}{m_1+m_2}=\frac{m_2(r_1-r_2)}{m_1+m_2} \\
r_b&=r_2-\frac{m_1r_1+m_2r_2}{m_1+m_2}=\frac{(m_1+m_2)r_2-m_1r_1-m_2r_2}{m_1+m_2}=\frac{m_1(r_2-r_1)}{m_1+m_2}
\end{align}

similarly, the velocity of center of mass is
\begin{align}
r_s=\frac{m_1v_1+m_2v_2}{m_1+m_2}
\end{align}

and the velocity of two particles refers to \textit{zero momentum frame} are
\begin{align}
v_a&=\frac{m_2(v_1-v_2)}{m_1+m_2} \\
v_b&=\frac{m_1(v_2-v_1)}{m_1+m_2}
\end{align}

then the momentum refers to \textit{zero momentum frame} are
\begin{align}
p_a=m_1v_a&=\frac{m_1m_2(v_1-v_2)}{m_1+m_2} \\
p_b=m_2v_b&=\frac{m_1m_2(v_2-v_1)}{m_1+m_2}
\end{align}

with reduced mass
\begin{align}
\mu=\frac{m_1+m_2}{m_1m_2}
\end{align}

we get
\begin{align}
p_a&=\frac{v_1-v_2}{\mu} \\
p_b&=\frac{v_2-v_1}{\mu}
\end{align}

now we take equation (7)
\begin{align}
\frac{1}{2}p_av_a+\frac{1}{2}p_bv_b&=\frac{1}{2}p_a'v_a'+\frac{1}{2}p_b'v_b'+Q=E \\
\frac{p^2_a}{m_1}+\frac{p^2_b}{m_2}&=\frac{p'^2_a}{m_1}+\frac{p'^2_b}{m_2}+2Q \\
\frac{1}{\mu}p^2_a&=\frac{1}{\mu}p'^2_a+2Q \\
\frac{1}{\mu}p'^2_a&=\frac{1}{\mu}p^2_a-2Q \\
p'^2_a&=p^2_a-2\mu Q=\frac{v_1^2-2v_1v_2+v_2^2}{\mu^2}-2\mu Q
\end{align}

\end{enumerate}

\end{document}
