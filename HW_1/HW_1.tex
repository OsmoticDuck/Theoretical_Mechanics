\documentclass[11pt,a4paper]{article}
\usepackage[utf8]{inputenc}
\usepackage[english]{babel}
\usepackage{fontspec}
\usepackage{xunicode}
\usepackage{xltxtra}
\usepackage{polyglossia}
\usepackage{amsmath}
\usepackage{amsfonts}
\usepackage{amssymb}
\usepackage{graphicx}
\usepackage[left=2cm,right=2cm,top=2cm,bottom=2cm]{geometry}
\usepackage{afterpage}
\usepackage{xcolor}
\usepackage{pagecolor}
\usepackage{mathtools}
%\usepackage{fancyhdr}
\usepackage{hyperref}
\usepackage{blindtext}
\usepackage{mahjong}
\usepackage{amsthm}
\usepackage{chngcntr}
\usepackage{tikz}
\usepackage{bbold}
\usepackage{pdfpages}
\usepackage{xfrac}

\usepackage{stmaryrd}
\let\stmaryrdLightning\lightning

\author{Zehao Gao}

\begin{document}

\section*{Homework 1 solutions}

\paragraph{Question 1}

\begin{enumerate}

\item[]

An elliptic curve is given by
\begin{align}
\left\{
\begin{array}{lcl}
x=a C_t \\
y=b S_t \\
z=0
\end{array}
\right.
\hspace{1em}
\mbox{or}
\hspace{1em}
\vec{r}(t)=
\begin{pmatrix}
a C_t \\
b S_t \\
0
\end{pmatrix}
\end{align}

we are going to find the TNB-frame of this curve.

First we determine the infinitesimal arc length $ds$
\begin{align}
ds
=\Big|\frac{\vec{r}(t)}{dt}\Big|dt
=\sqrt{(-a S_t)^2+(b C_t)^2}dt
=\sqrt{a^2+b^2}dt
=v(t)dt
\end{align}
where $v(t)=\sqrt{a^2+b^2}$ is a constant.

According to Frenet–Serret formulas, the tangential unit vector is given by
\begin{align}
\mathbf{T}
=\frac{d\vec{r}}{ds}
=\frac{d\vec{r}}{dt}\frac{dt}{ds}
=
\begin{pmatrix}
\frac{-a S_t}{\sqrt{a^2+b^2}} \\
\frac{b C_t}{\sqrt{a^2+b^2}} \\
0
\end{pmatrix}
\end{align}

and the normal unit vector is
\begin{align}
\mathbf{N}
=\frac{\frac{d\mathbf{T}}{ds}}{|\frac{d\mathbf{T}}{ds}|}
=\frac{\frac{d\mathbf{T}}{dt}\frac{dt}{ds}}{|\frac{d\mathbf{T}}{dt}\frac{dt}{ds}|}
\end{align}

where
\begin{align}
\frac{d\mathbf{T}}{dt}
=
\begin{pmatrix}
\frac{-a C_t}{\sqrt{a^2+b^2}} \\
\frac{-b S_t}{\sqrt{a^2+b^2}} \\
0
\end{pmatrix}
\end{align}

therefore
\begin{align}
\frac{d\mathbf{T}}{dt}\frac{dt}{ds}
=
\begin{pmatrix}
\frac{-a C_t}{a^2+b^2} \\
\frac{-b S_t}{a^2+b^2} \\
0
\end{pmatrix},
\hspace{1em}
\big|\frac{d\mathbf{T}}{dt}\frac{dt}{ds}\big|=
\sqrt{\frac{1}{a^2+b^2}}
\end{align}
\begin{align}
\mathbf{N}=
\begin{pmatrix}
\frac{-a C_t}{\sqrt{a^2+b^2}} \\
\frac{-b S_t}{\sqrt{a^2+b^2}} \\
0
\end{pmatrix}
\end{align}

the binormal  unit vector is given by
\begin{align}
\mathbf{B}=\mathbf{T}\times\mathbf{N}=
\begin{pmatrix}
0 \\
0 \\
\frac{ab S_{2t}}{a^2+b^2}
\end{pmatrix}
\end{align}

\newpage

The velocity and acceleration vector is given by
\begin{align}
\vec{v}(t)=
\begin{pmatrix}
-a S_t \\
b C_t \\
0
\end{pmatrix}
,\hspace{1em}
\vec{a}(t)=
\begin{pmatrix}
-a C_t \\
-b S_t \\
0
\end{pmatrix}
=a(t)\mathbf{T}+\frac{v^2(t)}{R(t)}\mathbf{N}
\end{align}
where $a(t)=0$ due to constant velocity

Therefore
\begin{align}
\begin{pmatrix}
-a C_t \\
-b S_t \\
0
\end{pmatrix}
=
\frac{a^2+b^2}{R}
\begin{pmatrix}
\frac{-a C_t}{\sqrt{a^2+b^2}} \\
\frac{-b S_t}{\sqrt{a^2+b^2}} \\
0
\end{pmatrix}
\end{align}
\begin{align}
R=\sqrt{a^2+b^2}
\end{align}

\end{enumerate}

\newpage

\paragraph{Question 2}

\begin{enumerate}

\item[]

A helix is given by
\begin{align}
\vec{r}(t)=
\begin{pmatrix}
\rho C_{\omega t} \\
\rho S_{\omega t} \\
\frac{1}{2}at^2
\end{pmatrix}
\end{align}

\begin{align}
ds
=\Big|\frac{\vec{r}(t)}{dt}\Big|dt
=\sqrt{\rho^2\omega^2+a^2t^2}dt
\end{align}

\begin{align}
\mathbf{T}
=\frac{d\vec{r}}{ds}
=\frac{d\vec{r}}{dt}\frac{dt}{ds}
=
\begin{pmatrix}
\rho C_{\omega t} \\
\rho S_{\omega t} \\
\frac{1}{2}at^2
\end{pmatrix}
\frac{1}{\sqrt{\rho^2\omega^2+a^2t^2}}
\end{align}

\begin{align}
\mathbf{N}
=\frac{\frac{d\mathbf{T}}{ds}}{|\frac{d\mathbf{T}}{ds}|}
=\frac{\frac{d\mathbf{T}}{dt}\frac{dt}{ds}}{|\frac{d\mathbf{T}}{dt}\frac{dt}{ds}|}
\end{align}

\begin{align}
\frac{d\mathbf{T}}{dt}
=
\begin{pmatrix}
\rho(\omega S_{\omega t}(\rho^2\omega^2+a^2t^2)+a^2t C_{\omega t}) \\
\rho(\omega C_{\omega t}(\rho^2\omega^2+a^2t^2\omega)-a^2t S_{\omega t}) \\
\frac{1}{2}a^3t^3+a\rho^2t\omega^2
\end{pmatrix}
\frac{1}{(\rho^2\omega^2+a^2t^2)^{\sfrac{3}{2}}}
\end{align}

\begin{align}
\frac{d\mathbf{T}}{dt}\frac{dt}{ds}
=
\begin{pmatrix}
\rho(\omega S_{\omega t}(\rho^2\omega^2+a^2t^2)+a^2t C_{\omega t}) \\
\rho(\omega C_{\omega t}(\rho^2\omega^2+a^2t^2\omega)-a^2t S_{\omega t}) \\
\frac{1}{2}a^3t^3+a\rho^2t\omega^2
\end{pmatrix}
\frac{1}{(\rho^2\omega^2+a^2t^2)^2}
\end{align}

\begin{align}
\big|\frac{d\mathbf{T}}{dt}\frac{dt}{ds}\big|
=
\sqrt{\bigg(\frac{a^3t^3}{2}+a\rho^2t\omega^2\bigg)^2+\rho^2(\omega C_{\omega t}(a^2t^2\omega+\rho^2\omega^2)-a^2t S_{\omega t})^2+\rho^2(\omega S_{\omega t}(a^2t^2+\rho^2\omega^2)+a^2t C_{\omega t})^2}
\end{align}

\end{enumerate}

\newpage

\paragraph{Question 3}

\begin{enumerate}

\item[(a)]

Prove the following identity
\begin{align}
\varepsilon_{ijk}\varepsilon_{lmk}=\delta_{il}\delta_{jm}-\delta_{im}\delta_{jl}
\end{align}

\begin{proof}

\begin{align}
\varepsilon_{ijk}\varepsilon_{lmk}=
\left\{
\begin{array}{lcl}
1 & \mbox{for} & i=l,j=m \\
-1 & \mbox{for} & i=m,j=l
\end{array}
\right.
\end{align}
where $i\neq j\neq k$ and $l\neq m\neq k$

\begin{align}
\delta_{il}\delta_{jm}-\delta_{im}\delta_{jl}=
\left\{
\begin{array}{lcl}
\delta_{ll}\delta_{mm}-\delta_{lm}\delta_{ml}=1-0=1 & \mbox{for} & i=l,j=m \\
\delta_{ml}\delta_{lm}-\delta_{mm}\delta_{ll}=0-1=-1 & \mbox{for} & i=m,j=l
\end{array}
\right.
\end{align}
where $i\neq j$ and $l\neq m$

Consider the third case, when $i=j=l=m$, we have
\begin{align}
\varepsilon_{ijk}\varepsilon_{lmk}=0
\end{align}
and
\begin{align}
\delta_{il}\delta_{jm}-\delta_{im}\delta_{jl}=1-1=0
\end{align}

\end{proof}

\item[(b)]

Prove the following identity
\begin{align}
(\vec{a}\times\vec{b})_i=\varepsilon_{ijk}a_jb_k
\end{align}

\begin{proof}

\begin{align}
\vec{a}\times\vec{b}=
\begin{pmatrix}
a_i \\
a_j \\
a_k
\end{pmatrix}
\times
\begin{pmatrix}
b_i \\
b_j \\
b_k
\end{pmatrix}
=
\begin{pmatrix}
a_jb_k-a_kb_j \\
a_kb_i-a_ib_k \\
a_ib_j-a_jb_i
\end{pmatrix}
=
\begin{pmatrix}
(\vec{a}\times\vec{b})_i \\
(\vec{a}\times\vec{b})_j \\
(\vec{a}\times\vec{b})_k
\end{pmatrix}
\end{align}
therefore
\begin{align}
(\vec{a}\times\vec{b})_i=a_jb_k-a_kb_j
\end{align}

\begin{align}
\varepsilon_{ijk}=
\left\{
\begin{array}{lcl}
1 & \mbox{for} & j\neq k\neq i \\
-1 & \mbox{for} & j,k\mbox{ inverse}(\varepsilon_{ikj}) \\
0 & \mbox{for} & (j=k)\vee (j=i) \vee (k=i)
\end{array}
\right.
\end{align}
therefore
\begin{align*}
\varepsilon_{ijk}a_jb_k=a_jb_k-a_kb_j
\end{align*}

\end{proof}

\end{enumerate}

\newpage

\paragraph{Question 4}

\begin{enumerate}

\item[(a)]

A field is given by
\begin{align}
\vec{F}=\bigg(\frac{-y}{r^2},\frac{x}{r^2},0\bigg)\cdot\beta
\end{align}

To verify if the field is conservative, we apply $\nabla\times$
\begin{align}
\nabla\times\vec{F}=\beta\nabla\times\bigg(\frac{-y}{r^2},\frac{x}{r^2},0\bigg)=\beta\nabla\times\vec{F}'
\end{align}

since
\begin{align}
(\nabla\times\vec{F})_i=\epsilon_{ijk}\partial_j F_k
\end{align}

we get
\begin{align}
\nabla\times\vec{F}'=
\begin{pmatrix}
\partial_y\vec{F}'_z-\partial_z\vec{F}'_y \\
\partial_z\vec{F}'_x-\partial_x\vec{F}'_z \\
\partial_y\vec{F}'_x-\partial_x\vec{F}'_y
\end{pmatrix}=
\begin{pmatrix}
0 \\
0 \\
\frac{-1}{x^2+y^2}+\frac{2y^2}{(x^2+y^2)^2}-\frac{1}{x^2+y^2}+\frac{2x^2}{(x^2+y^2)^2}
\end{pmatrix}=0
\end{align}

the field is curl-free.
\item[(b)]

A circle path is given by
\begin{align}
\vec{r}=
\begin{pmatrix}
C_t \\
S_t
\end{pmatrix}
\end{align}

calculate the line integral
\begin{align}
\oint_C \vec{F}\cdot d\vec{l}
\end{align}

infinitesimal line element
\begin{align}
dl=\bigg|\frac{d\vec{r}}{dt}\bigg|dt=dt
\end{align}

therefore
\begin{align}
\mathbf{T}=\frac{d\vec{r}}{dt}=
\begin{pmatrix}
-S_t \\
C_t
\end{pmatrix}
\end{align}
\begin{align}
d\vec{l}=dt\mathbf{T}=
\begin{pmatrix}
-S_tdt \\
C_tdt
\end{pmatrix}
\end{align}

therefore
\begin{align}
\oint_C \vec{F}\cdot d\vec{l}=\int_0^{2\pi}\vec{F}(r(t))\cdot\mathbf{T}dt=
\beta\int_0^{2\pi}
\begin{pmatrix}
-S_t \\
C_t
\end{pmatrix}\cdot
\begin{pmatrix}
-S_t \\
C_t
\end{pmatrix}dt
=
\beta\int_0^{2\pi}dt=2\pi\beta
\end{align}

\end{enumerate}

\end{document}
