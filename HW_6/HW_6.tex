\documentclass[11pt,a4paper]{article}
\usepackage[utf8]{inputenc}
\usepackage[english]{babel}
\usepackage{fontspec}
\usepackage{xunicode}
\usepackage{xltxtra}
\usepackage{polyglossia}
\usepackage{amsmath}
\usepackage{amsfonts}
\usepackage{amssymb}
\usepackage{graphicx}
\usepackage[left=2cm,right=2cm,top=2cm,bottom=2cm]{geometry}
\usepackage{afterpage}
\usepackage{xcolor}
\usepackage{pagecolor}
\usepackage{mathtools}
%\usepackage{fancyhdr}
\usepackage{hyperref}
\usepackage{blindtext}
\usepackage{mahjong}
\usepackage{amsthm}
\usepackage{chngcntr}
\usepackage{tikz}
\usepackage{bbold}
\usepackage{pdfpages}
\usepackage{xfrac}

\usepackage{stmaryrd}
\let\stmaryrdLightning\lightning

\newcommand{\tensor}{\stackrel{\leftrightarrow}}

\author{Zehao Gao}

\begin{document}

\section*{Homework 6 solutions}

\paragraph{Question 1}

\begin{proof}

\begin{align}
\Theta_{kl}=\sum_{\nu=1}^N m_\nu(r_\nu'^2\delta_{kl}-x_{\nu k}'x_{\nu l}')=\sum_{\nu=1}^N m_\nu(r_\nu'^2\delta_{lk}-x_{\nu l}'x_{\nu k}')=\Theta_{lk}
\end{align}

\end{proof}

\paragraph{Question 2}

A homogeneous ellipsoid is given by
\begin{align}
\frac{x^2}{a^2}+\frac{y^2}{b^2}+\frac{z^2}{c^2}=1
\end{align}
with mass density $\rho$
\begin{align}
\Theta_{kl}=\int dV \rho (r^2\delta_{kl}-x_k x_l)
\end{align}
therefore
\begin{align}
\Theta_{xx}&=\int dV \rho (y^2+z^2) \\
\Theta_{yy}&=\int dV \rho (x^2+z^2) \\
\Theta_{zz}&=\int dV \rho (x^2+y^2)
\end{align}
In spherical coordinates the ellipsoid can be written as
\begin{align}
\frac{r^2 C^2_\vartheta S^2_\varphi}{a^2}+
\frac{r^2 S^2_\vartheta S^2_\varphi}{b^2}+
\frac{r^2 C^2_\varphi}{c^2}=1
\end{align}
define
\begin{align}
x'&=ax=ar^2 C_\vartheta S_\varphi \\
y'&=by=br^2 S_\vartheta S_\varphi \\
z'&=cz=cr^2 C_\varphi
\end{align}
therefore the infinitesimal volume element $dV$ is to be multiplied with a coefficient $abc$ resulting $dV'=abc dV=abcr^2 S_\vartheta drd\varphi d\vartheta$.
%\begin{align}
%r=\sqrt{\frac{1}{\frac{C^2_\vartheta S^2_\varphi}{a^2}+\frac{S^2_\vartheta S^2_\varphi}{b^2}+\frac{C^2_\varphi}{c^2}}}
%\end{align}
Then integrals from equation 4, 5 and 6 are to calculated.
\begin{align}
\Theta_{xx}
&=8\rho abc \int_0^{\frac{\pi}{2}} d\vartheta \int_0^{\frac{\pi}{2}} d\varphi \int_0^1 dr
(r^2 S^2_\vartheta S^2_\varphi+r^2 C^2_\varphi)r^2 S_\vartheta \\
&=8\rho abc \int_0^{\frac{\pi}{2}} d\vartheta \int_0^{\frac{\pi}{2}} d\varphi \int_0^1 dr
(r^4 S^3_\vartheta S^2_\varphi+r^4 S_\vartheta C^2_\varphi) \\
&=\frac{8\rho abc}{5} \int_0^{\frac{\pi}{2}} d\vartheta \int_0^{\frac{\pi}{2}} d\varphi
(S^3_\vartheta S^2_\varphi+S_\vartheta C^2_\varphi) \\
&=\lightning
\end{align}

\newpage

\paragraph{Question 3}

\begin{enumerate}

\item[(a)]

\begin{align}
\Theta_{kl}=\sum_{\nu=1}^N m_\nu(r_\nu'^2\delta_{kl}-x_{\nu k}'x_{\nu l}')
\end{align}

\begin{center}%tikzpic

\tikzset{every picture/.style={line width=0.75pt}} %set default line width to 0.75pt        

\begin{tikzpicture}[x=0.75pt,y=0.75pt,yscale=-1,xscale=1]
%uncomment if require: \path (0,300); %set diagram left start at 0, and has height of 300

%Straight Lines [id:da7009131457982225] 
\draw    (196.5,158) -- (145.19,209.31) ;
\draw [shift={(143.78,210.72)}, rotate = 315] [color={rgb, 255:red, 0; green, 0; blue, 0 }  ][line width=0.75]    (10.93,-3.29) .. controls (6.95,-1.4) and (3.31,-0.3) .. (0,0) .. controls (3.31,0.3) and (6.95,1.4) .. (10.93,3.29)   ;
%Straight Lines [id:da5098527902445784] 
\draw    (196.5,158) -- (305.16,158) ;
\draw [shift={(307.16,158)}, rotate = 180] [color={rgb, 255:red, 0; green, 0; blue, 0 }  ][line width=0.75]    (10.93,-3.29) .. controls (6.95,-1.4) and (3.31,-0.3) .. (0,0) .. controls (3.31,0.3) and (6.95,1.4) .. (10.93,3.29)   ;
%Straight Lines [id:da35952076475793926] 
\draw    (196.5,158) -- (196.5,51.72) ;
\draw [shift={(196.5,49.72)}, rotate = 90] [color={rgb, 255:red, 0; green, 0; blue, 0 }  ][line width=0.75]    (10.93,-3.29) .. controls (6.95,-1.4) and (3.31,-0.3) .. (0,0) .. controls (3.31,0.3) and (6.95,1.4) .. (10.93,3.29)   ;
%Shape: Cube [id:dp1396227675878965] 
\draw   (167.5,118.99) -- (196.43,90.06) -- (264.16,90.06) -- (264.16,157.57) -- (235.22,186.5) -- (167.5,186.5) -- cycle ; \draw   (264.16,90.06) -- (235.22,118.99) -- (167.5,118.99) ; \draw   (235.22,118.99) -- (235.22,186.5) ;

% Text Node
\draw (266,116.5) node [anchor=north west][inner sep=0.75pt]   [align=left] {a};
% Text Node
\draw (198.5,186) node [anchor=north west][inner sep=0.75pt]   [align=left] {a};
% Text Node
\draw (152.5,107) node [anchor=north west][inner sep=0.75pt]   [align=left] {m};

\end{tikzpicture}
\end{center}

We have 8 particles in total
\begin{align*}
&P_1(0,0,0) \\
&P_2(a,0,0) \\
&P_3(0,a,0) \\
&P_4(0,0,a) \\
&P_5(a,a,0) \\
&P_6(a,0,a) \\
&P_7(0,a,a) \\
&P_8(a,a,a)
\end{align*}
the inertia tensor is therefore given by
\begin{align}
\Theta_{xx}
&=m\sum_{\nu=1}^N (r_\nu'^2-x^2) \\
&=m(
0+
0+
a^2+
a^2+
a^2+
a^2+
2a^2+
2a^2
) \\
&=8a^2m=\Theta_{yy}=\Theta_{zz} \\
\Theta_{xy}
&=m\sum_{\nu=1}^N (-xy) \\
&=m(
-0
-0
-0
-0
-a^2
-0
-0
-a^2
) \\
&=-2a^2m=\Theta_{xz}=\Theta_{zy}
\end{align}
\begin{align}
\tensor{\Theta}=a^2m
\begin{pmatrix}
8 & -2 & -2 \\
-2 & 8 & -2 \\
-2 & -2 & 8
\end{pmatrix}
\end{align}

\newpage

\item[(b)]

Consider a situation that four particles in the top layer have a mass of $m$, while particles in the bottom layer have $2m$.
\begin{align*}
&P_1(0,0,0),2m \\
&P_2(a,0,0),2m \\
&P_3(0,a,0),2m \\
&P_4(0,0,a),m \\
&P_5(a,a,0),2m \\
&P_6(a,0,a),m \\
&P_7(0,a,a),m \\
&P_8(a,a,a),m
\end{align*}
the inertia tensor is therefore given by
\begin{align}
\Theta_{xx}
&=m\sum_{\nu=1}^N (r_\nu'^2-x^2) \\
&=
2m(
0+
0+
a^2+
a^2)+
m(
a^2+
a^2+
2a^2+
2a^2
) \\
&=10a^2m=\Theta_{yy}\\
\Theta_{zz}
&=m\sum_{\nu=1}^N (r_\nu'^2-z^2) \\
&=
2m(
0+
a^2+
a^2+
2a^2)+
m(
0+
a^2+
a^2+
2a^2
) \\
&=12a^2m \\
\Theta_{xy}
&=m\sum_{\nu=1}^N (-xy) \\
&=
2m(
-0
-0
-0
a^2)+
m(
-0
-0
-0
-a^2
) \\
&=-3a^2m \\
\Theta_{xz}
&=m\sum_{\nu=1}^N (-xz) \\
&=
2m(
-0
-0
-0
-0)+
m(
-0
-a^2
-0
-a^2
) \\
&=-2a^2m=\Theta_{yz}
\end{align}
\begin{align}
\tensor{\Theta}=a^2m
\begin{pmatrix}
10 & -3 & -2 \\
-3 & 10 & -2 \\
-2 & -2 & 12
\end{pmatrix}
\end{align}

\end{enumerate}

\end{document}
